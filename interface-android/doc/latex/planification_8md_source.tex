\hypertarget{planification_8md_source}{}\subsection{planification.\+md}

\begin{DoxyCode}
00001 # Planification
00002 
00003 ## Liste des fonctionnalitées attendues
00004 ---
00005 ### Partie IR : MHADI & FOUCAULT
00006 ---
00007 -     Affichage des mesures des capteurs (température int. & ext., humidité int. & ext., pression
       atmosphérique, poids, ensoleillement, localisation et niveau de charge)
00008 -     Recevoir les données des capteurs
00009 -     Gestion de la ruche : Nomenclature / Paramètrage / Ajout-Supression
00010 -     Alertes via mails notifications, mails et sms
00011 -     Generation d'un graphique à partir des mesures
00012 -     Actualisation des données en temps réels (horodatage)
00013 -     Création interface PC
00014 -     Création interface Mobile
00015 
00016 ---
00017 ### Partie EC : PERRIN
00018 ---
00019 -     Installation des capteurs (température int. & ext., humidité int. & ext., pression
       atmosphérique, poids, ensoleillement, localisation et niveau de charge)
00020 -     Mise en place de la Jauge de contrainte
00021 -     Alimentation de la carte Lora et connection de celle-ci aux différents capteurs
00022 -     Envoie des données au serveur
00023 
00024 ---
00025 ---
00026 
00027 ## Classification des fonctionnalitées attendues
00028 
00029 Fonctionnalités IR                                                               |  Priorité  |
00030 ---------------------------------------------------------------------------------|------------|
00031 Création interface PC                                                            |  Haute     |
00032 Création interface Mobile                                                        |  Haute     |
00033 Affichage des mesures des capteurs (température int. & ext., humidité int. & ext., pression
       atmosphérique, poids, ensoleillement, localisation et niveau de charge)                                               
                                 |  Haute     |
00034 Recevoir les données des capteurs                                                |  Haute     |
00035 Gestion de la ruche : Nomenclature / Paramètrage / Ajout-Supression              |  Moyenne   |
00036 Generation d'un graphique à partir des mesures                                   |  Moyenne   |
00037 Actualisation des données en temps réels (horodatage)                            |  Moyenne    |
00038 Alertes via mails notifications, mails et sms                                    |  Faible    |
00039 
00040 Fonctionnalités EC                                                               |  Priorité  |
00041 ---------------------------------------------------------------------------------|------------|
00042 Installation des capteurs (température int. & ext., humidité int. & ext., pression atmosphérique,
       poids, ensoleillement, localisation et niveau de charge)                         |  Haute     |
00043 Mise en place de la Jauge de contrainte                                          |  Haute     |
00044 Alimentation de la carte Lora et connection de celle-ci aux différents capteurs  |  Haute     |
00045 Envoie des données au serveur                                                    |  Haute     |
00046 
00047 ---
00048 ---
00049 
00050 ## Répartition des fonctionnalités attendues
00051 
00052 Fonctionnalités IR                                                               | Itération  |
00053 ---------------------------------------------------------------------------------|------------|
00054 Création interface PC                                                            |  1         |
00055 Création interface Mobile                                                        |  1         |
00056 Affichage des mesures des capteurs (température int. & ext., humidité int. & ext., pression
       atmosphérique, poids, ensoleillement, localisation et niveau de charge)                                               
                                 |  1         |
00057 Recevoir les données des capteurs                                                |  1         |
00058 Gestion de la ruche : Nomenclature / Paramètrage / Ajout-Supression              |  2         |
00059 Generation d'un graphique à partir des mesures                                   |  2         |
00060 Alertes via mails notifications, mails et sms                                    |  3         |
00061 Actualisation des données en temps réels (horodatage)                            |  3         |
00062 
00063 Fonctionnalités EC                                                               | Itération  |
00064 ---------------------------------------------------------------------------------|------------|
00065 Installation des capteurs (température int. & ext., humidité int. & ext., pression atmosphérique,
       poids, ensoleillement, localisation et niveau de charge)                         |  1         |
00066 Mise en place de la Jauge de contrainte                                          |  1         |
00067 Alimentation de la carte Lora et connection de celle-ci aux différents capteurs  |  1         |
00068 Envoie des données au serveur                                                    |  1         |
\end{DoxyCode}
